\documentclass[a4paper,10pt]{article}
\usepackage{epsfig}
\usepackage{latexsym}
\usepackage{graphicx}
\usepackage{amsfonts}
\usepackage{amsmath}
\usepackage{natbib}
\usepackage{xcolor}
\usepackage{verbatim}
%
\topmargin=-3cm
\topmargin=-1cm
\oddsidemargin=-1cm
\evensidemargin=-1cm
\textwidth=17cm
%
\textheight=27cm
\textheight=25cm
\raggedbottom
\sloppy





% Bibliography and bibfile
\def\aj{AJ}%
          % Astronomical Journal
\def\actaa{Acta Astron.}%
          % Acta Astronomica
\def\araa{ARA\&A}%
          % Annual Review of Astron and Astrophys
\def\apj{ApJ}%
          % Astrophysical Journal
\def\apjl{ApJ}%
          % Astrophysical Journal, Letters
\def\apjs{ApJS}%
          % Astrophysical Journal, Supplement
\def\ao{Appl.~Opt.}%
          % Applied Optics
\def\apss{Ap\&SS}%
          % Astrophysics and Space Science
\def\aap{A\&A}%
          % Astronomy and Astrophysics
\def\aapr{A\&A~Rev.}%
          % Astronomy and Astrophysics Reviews
\def\aaps{A\&AS}%
          % Astronomy and Astrophysics, Supplement
\def\azh{AZh}%
          % Astronomicheskii Zhurnal
\def\baas{BAAS}%
          % Bulletin of the AAS
\def\bac{Bull. astr. Inst. Czechosl.}%
          % Bulletin of the Astronomical Institutes of Czechoslovakia 
\def\caa{Chinese Astron. Astrophys.}%
          % Chinese Astronomy and Astrophysics
\def\cjaa{Chinese J. Astron. Astrophys.}%
          % Chinese Journal of Astronomy and Astrophysics
\def\icarus{Icarus}%
          % Icarus
\def\jcap{J. Cosmology Astropart. Phys.}%
          % Journal of Cosmology and Astroparticle Physics
\def\jrasc{JRASC}%
          % Journal of the RAS of Canada
\def\mnras{MNRAS}%
          % Monthly Notices of the RAS
\def\memras{MmRAS}%
          % Memoirs of the RAS
\def\na{New A}%
          % New Astronomy
\def\nar{New A Rev.}%
          % New Astronomy Review
\def\pasa{PASA}%
          % Publications of the Astron. Soc. of Australia
\def\pra{Phys.~Rev.~A}%
          % Physical Review A: General Physics
\def\prb{Phys.~Rev.~B}%
          % Physical Review B: Solid State
\def\prc{Phys.~Rev.~C}%
          % Physical Review C
\def\prd{Phys.~Rev.~D}%
          % Physical Review D
\def\pre{Phys.~Rev.~E}%
          % Physical Review E
\def\prl{Phys.~Rev.~Lett.}%
          % Physical Review Letters
\def\pasp{PASP}%
          % Publications of the ASP
\def\pasj{PASJ}%
          % Publications of the ASJ
\def\qjras{QJRAS}%
          % Quarterly Journal of the RAS
\def\rmxaa{Rev. Mexicana Astron. Astrofis.}%
          % Revista Mexicana de Astronomia y Astrofisica
\def\skytel{S\&T}%
          % Sky and Telescope
\def\solphys{Sol.~Phys.}%
          % Solar Physics
\def\sovast{Soviet~Ast.}%
          % Soviet Astronomy
\def\ssr{Space~Sci.~Rev.}%
          % Space Science Reviews
\def\zap{ZAp}%
          % Zeitschrift fuer Astrophysik
\def\nat{Nature}%
          % Nature
\def\iaucirc{IAU~Circ.}%
          % IAU Cirulars
\def\aplett{Astrophys.~Lett.}%
          % Astrophysics Letters
\def\apspr{Astrophys.~Space~Phys.~Res.}%
          % Astrophysics Space Physics Research
\def\bain{Bull.~Astron.~Inst.~Netherlands}%
          % Bulletin Astronomical Institute of the Netherlands
\def\fcp{Fund.~Cosmic~Phys.}%
          % Fundamental Cosmic Physics
\def\gca{Geochim.~Cosmochim.~Acta}%
          % Geochimica Cosmochimica Acta
\def\grl{Geophys.~Res.~Lett.}%
          % Geophysics Research Letters
\def\jcp{J.~Chem.~Phys.}%
          % Journal of Chemical Physics
\def\jgr{J.~Geophys.~Res.}%
          % Journal of Geophysics Research
\def\jqsrt{J.~Quant.~Spec.~Radiat.~Transf.}%
          % Journal of Quantitiative Spectroscopy and Radiative Trasfer
\def\memsai{Mem.~Soc.~Astron.~Italiana}%
          % Mem. Societa Astronomica Italiana
\def\nphysa{Nucl.~Phys.~A}%
          % Nuclear Physics A
\def\physrep{Phys.~Rep.}%
          % Physics Reports
\def\physscr{Phys.~Scr}%
          % Physica Scripta
\def\planss{Planet.~Space~Sci.}%
          % Planetary Space Science
\def\procspie{Proc.~SPIE}%
          % Proceedings of the SPIE
\let\astap=\aap
\let\apjlett=\apjl
\let\apjsupp=\apjs
\let\applopt=\ao

\newcommand{\Fc}{{\cal F}}
\newcommand{\Xib}{\mbox{\boldmath $\Xi$}}

\title{EBEX `\texttt{hexsky}' pointing simulator documentation, v0.2} \author{Sam Leach}
\begin{document} \maketitle

\begin{abstract}
  The goal of this memo is to provide documentation for the
  balloon-borne scanning CMB experiment simulator
  \texttt{hexsky}. These codes allow to simulate the boresight and
  detector pointing from an EBEX `schedule file', automatically
  documenting the scanning strategy.
\end{abstract}


\tableofcontents

%\citet{2009arXiv0912.2338V} recently published a catalogue of

\section{Introduction}

This memo summarises the main features of the balloon-borne scanning
CMB experiment simulator known as the `\texttt{hexsky}' library. The goal
of these codes is to simulate the pointing from and validate the
content of EBEX \emph{schedule files}, which are list of pointing
commands for automated in-flight scanning. In this simulations
context, schedule files along with a \emph{mission file} are first
parsed by IDL, and then converted into parameter files that can be
used by the \texttt{hexsky} binary. Pointing is outputted into a dirfile
and then subsequently validated by automatically producing a document
which gives a high level overview of the scanning strategy.



\section{Installation}

In this section we reproduce the content of the
\texttt{hexsky/INSTALL} file which describes installation,
dependencies and gives and an example `quick start' run of the codes.

\verbatiminput{../INSTALL}


%% \begin{figure}
%% \begin{center}
%% \psfig{file=figures/12302011.png,width=5.5in}
%% \psfig{file=figures/1142012.png,width=5.5in}
%% \caption{EBEX visibility and calibrator plot. (Upper panel) Accessing
%%   Cen A from around January 1st 2012 onwards by relaxing the anti-sun
%%   constraint to $\pm 90^\circ$. (Lower panel) Between 14-18 January
%%   2012 Cen A is blocked by the moon.} \label{fig:ebex2}
%% \end{center}
%% \end{figure}


%% \scriptsize
%% \begin{tabular}{|c|cc|cc|cc|cc|}
%% \hline
%% %\tableline
%% Source		&90 GHz		&90 GHz		&150 GHz	&150 GHz	&240 GHz	&240 GHz	&400 GHz	&400 GHz\\
%% 		&1$\sigma_{b}$ (Jy)	&2$\sigma_{b}$ (Jy)	&1$\sigma_{b}$ (Jy)	&2$\sigma_{b}$ (Jy)	&1$\sigma_{b}$ (Jy)	&2$\sigma_{b}$ (Jy)	&1$\sigma_{b}$ (Jy)	&2$\sigma_{b}$ (Jy)\\
%% \hline
%% %\tableline
%% Carina		&72.4$\pm$8.2	&214$\pm19$	&37.9$\pm$4.1	&131$\pm$13	&132$\pm$18	&410$\pm$56	&446$\pm$145	&1324$\pm$419\\
%% RCW38		&60.8$\pm$6.2	&137$\pm$16	&51.7$\pm$4.0	&122$\pm$12	&126$\pm$17	&298$\pm$42	&406$\pm$115	&1002$\pm$298\\
%% IRAS1022	&60.7$\pm$4.6	&143$\pm$15	&42.2$\pm$2.9	&119$\pm$11	&107$\pm$15	&282$\pm$38	&320$\pm$94	&904$\pm$278\\
%% NGC3603		&55.6$\pm$6.1	&126$\pm$14	&42$\pm$1.2	&107$\pm$4.4	&101$\pm$14	&249$\pm$35	&308$\pm$99	&802$\pm$256\\
%% NGC3576		&32.6$\pm$3.6	&74.6$\pm$9.1	&32.7$\pm$1.5	&70.1$\pm$6.0	&87.9$\pm$10	&201$\pm$24	&360$\pm$102	&727$\pm$196\\
%% IRAS08576	&9.1$\pm$0.7	&20.6$\pm$2.6	&11.8$\pm$0.7	&31.9$\pm$2.2	&55.2$\pm$6.3	&158$\pm$18	&230$\pm$59	&667$\pm$173\\
%% 0537-441	&1.8$\pm$0.4	&3.8$\pm$1.1	&1.5$\pm$0.2	&3.1$\pm$0.6	&1.3$\pm$0.8	&2.3$\pm$2.1	&--	&--\\
%% 0521-365	&1.5$\pm$0.5	&2.9$\pm$1.4	&1.3$\pm$0.3	&2.9$\pm$0.7	&1.3$\pm$0.7	&2.6$\pm$0.4	&--	&--\\
%% 0438-43		&1.4$\pm$0.5	&3.2$\pm$1.3	&1.2$\pm$0.2	&2.9$\pm$0.6	&--		&--		&--	&--\\
%% \hline
%% %\tableline
%% \end{tabular}
%% \end{center}
%% \caption{\emph{(Table and caption take from
%% \citep{2003astro.ph..1599C}).} Fluxes in Jy of sources as determined
%% by observations with BOOMERANG. The 1$\sigma_{b}$ and 2$\sigma_{b}$
%% fluxes are the integrated fluxes within radii of 1 and 2
%% $\sigma_{beam}$ respectively. Errors are statistical only; they do not
%% include systematic errors due to uncertainty in the beam size,
%% background subtraction or calibration. For a point source observed
%% with a gaussian beam, the integrated fluxes out to 1$\sigma_{b}$ and
%% 2$\sigma_{b}$ are 0.393 and 0.865 of the source flux, respectively.
%% The extra-galactic sources are point sources, whereas the Galactic
%% sources are somewhat extended.}
%% \label{tbl2}
%% \end{table*}


\section{Library directory structure}

This section is a brief inventory of the \texttt{hexsky} library
in order to provide orientation towards some of the codes.
\\
\\
{\bf \texttt{bin/}} - Contains binaries. The binary of the \texttt{hexsky} C
code is placed here.
\\
\\
{\bf \texttt{data/}} - Contains various small bits of plain text data,
including some instrument bandpasses, locations of detectors in the
focalplane ({\texttt ebex\_fpdb.txt}), several source catalogs. In
addition it contains the \texttt{getdata.sh} script for downloading
larger files (Healpix foreground maps).
\\
\\
{\bf \texttt{missionfiles/}} - Contains plain text `mission files'.
\\
\\
{\bf \texttt{parfiles/}} - Contains example parameter files for running the
\texttt{hexsky} binary in standalone mode. The parameter file \texttt{example\_ldb.par}
has the privileged role of being the `template' parameter file, as defined in
\texttt{pro/hexsky/hexskylib.pro}.
\\
\\
{\bf \texttt{pro/}} - Contains the IDL library of the author. The guts of the scanning related
codes are in \texttt{pro/hexsky}, \texttt{pro/ebexcommanding}, \texttt{pro/commandfile\_interface},
\texttt{pro/astro}, and \texttt{pro/ebexfp}.  
\\
\\
{\bf \texttt{python/}} - Contains python codes. Currently there is just \texttt{batterySimulation.py}
code written by Alexander Smith and Britt Reichborn-Kjennerud (Columbia).
\\
\\
{\bf \texttt{schedulefiles/}} - Contains the \texttt{.sch} schedule files.
\\
\\
{\bf \texttt{src/}} - Contains the makefile and source code for the \texttt{hexsky} binary.
\\
\\
{\bf \texttt{strategyfiles/}} - Contains the \texttt{.ess} (EBEX
scanning strategy) files. These are parameters files needed for producing a
scanning strategy with \texttt{suggest\_scanning\_strategy.pro}

\section{Inputs}

The three inputs to a simulation run are a \emph{mission file}, a
\emph{schedule file} and a \emph{focal plane file}.

\begin{itemize}
\item The mission file contains free parameters relating to the mission:
Launch date, launch location orbit speed, sample rate, elevation
constraints are a few notable examples. Additional parameters can be
entered into the mission file and will be read into a mission IDL
struct without need to recode the parameter file parsing. The mission
file is parsed by the high-level main driver routine
\texttt{pro/hexsky/hexsky\_test\_driver.pro}.

\item The schedule file encodes the scanning strategy in the form of a
list of scanning commands that referenced to a specific time given in
the first line of the file. The schedule file is first parsed in
the \texttt{pro/hexsky/hexsky\_test\_driver.pro} driver and stored
using the schedule and command struct defined in
\texttt{pro/commandfile\_interface/schedule\_\_define.pro} and
\texttt{pro/commandfile\_interface/command\_\_define.pro}
respectively. After parsing the schedule file is converted into
parameter files that can are used for executing the \texttt{hexsky}
binary. This conversion is performed in
\texttt{pro/hexsky/simulate\_pointing.pro}, which must be maintained
along with the two files
\texttt{pro/ebexcommanding/get\_command\_list.pro} and
\texttt{pro/ebexcommanding/get\_scan\_period\_estimate.pro} in order
to simulate new or modified commands.

An explicit definition of the schedule file will be added to this
document in future revisions. Note that the \texttt{schedulefiles/}
directory contains a mixture of `UTC' schedule files and `LST'
schedule files. The LST schedule files are the only ones can be used
the flight computer, with each command's two timing parameters (day
and hour) given in local sidereal time, as well as using a special
convention for the reference time in the first line of the schedule file.
The UTC schedule files are a
standard `internal' to the \texttt{hexsky} library in which the
commands timing parameters are given in UTC timing (for subsequent use in the
\texttt{hexsky} binary pointing simulator parameter files). Tools for converting between
UTC and LST schedule files are found in \texttt{pro/commandfile\_interface/}.

\item The focal plane file contains information about the locations of
detectors in the focal plane, and classifies the detectors according
to the channel to which they belong. The current focal plane file is
given by \texttt{data/ebex\_fpdb.txt}. It is first parsed in the
\texttt{pro/hexsky/hexsky\_test\_driver.pro} by the
\texttt{pro/ebexfp/read\_focalplanefile.pro} code and stored in an
array of detector structs, defined in
\texttt{pro/ebexfp/detector\_\_define.pro}.


\end{itemize}


\section{Outputs}

In this section we list the main output from a simulation run. By
default, these will all be found in the \texttt{output/} directory of
the working directory. Some of the output files will have filenames
partly composed by the fileroot of the schedule file filename.


\begin{itemize}
\item \texttt{command\_firstsample.dat} - Plain text file containing
  the first sample and nsample for each command in the slow dirfile.

\item \texttt{data/} - Column data describing the survey geomety and
  depth ($f_{sky}$ vs ($\mu$K.arcmin)$^{-1}$). Arranged by channel subdirectories
  \texttt{boresight/}, \texttt{150/}, \texttt{250/}, and \texttt{410/}.



\item \texttt{dirfile/} - This is main pointing output from the
\texttt{hexsky} binary. The available fields are AZ, EL, RA, DEC, BETA
(orientation), ROLL (roll pendulation, to be added to BETA), LON, LAT,
LST (local sidereal time in hours) and RJD (reduced Julian day, where
the offset to be added, RJD0$=$2454900, is defined both in
\texttt{pro/hexsky/defineconstants.pro} and in
\texttt{src/utilities\_pointing.h}. In addition the fields AZ\_SUN,
EL\_SUN, AZ\_MOON, and EL\_MOON are added in `post-processing' of the dirfile
by \texttt{pro/hexsky/hexsky\_write\_sunazel\_fields.pro}. The
sample rate of this dirfile is set in the mission file with the
parameter \texttt{tsamp\_sec}, which will also control the overall run
time of the codes.

\item \texttt{doc/} - Documentation (postscript files) arranged by
  channel subdirectories \texttt{boresight/}, \texttt{150/},
  \texttt{250/}, and \texttt{410/}. This documentation is produced by
  \texttt{pro/hexsky/get\_scanning\_strategy\_documentation.pro}

\item \texttt{fp/} - In the case where a focal plane rotation is
  desired (using the parameter \texttt{focalplane\_phi0\_deg} in
  \texttt{hexsky\_test\_driver.pro}) the focal plane data is found in
  this directory.

\item \texttt{hexsky\_fullsamplerate\_tasks.txt} - This is a script to
which can be run in order to produce a separate dirfile
(\texttt{fullsamplerate/dirfile}) with boresight pointing at the full
sample rate, as defined by the parameter \texttt{tsamp\_fast\_sec} in
the mission file. It only depends on being able to run the
\texttt{hexsky} binary code and corresponding full sample rate
parameter files found in the \texttt{parfiles/} directory.

\item \texttt{hexsky\_outfiles/} - Various small pieces of output data
  from the \texttt{hexsky} binary including the
  \texttt{command*\_azel\_repointing.dat} files which contain az/el
  repointings, used by \texttt{plot\_schedule\_azel.pro}.

\item \texttt{logfiles/} - Output logfiles containing the standard
  output from the \texttt{hexsky} binary.


\item \texttt{maps/} - Output Healpix maps by channel subdirectories
  \texttt{boresight/}, \texttt{150/}, \texttt{250/}, and \texttt{410/}.

\begin{itemize}

\item \texttt{tint\_*.fits} - Healpix maps containing the total
  integration time, in Celestial coordinates. The map
  \texttt{tint\_bycommand*.fits} contains several columns, one
  corresponding to each type EBEX command \texttt{(cmb\_scan},
  \texttt{calibrator\_scan} etc).

\item \texttt{visibility\_*.fits} - A Healpix map containing a `sky
visibility map' showing which portion of the sky is visible (and for
how many hours over a 24 hour period) from the site defined in the mission
file, at the start of the first command.

\end{itemize}

\begin{itemize}

\item \texttt{scanning\_doc\_*.pdf} - This is a high level and command
by command latex document in PDF format summarising of the scanning
strategy. The main driver routine that produces this document one the
dirfile pointing has been generated is
\texttt{pro/hexsky/get\_scanning\_strategy\_documentation.pro}.

\end{itemize}

\item \texttt{parfiles/} - Parameter files used by the \texttt{hexsky}
  binary both at slow and full sample rate.

\item \texttt{ps/} - Contains all the postscript figures that are used in the

\item \texttt{*\_utc.sch} - This is a `UTC schedule file' (an internal
format for simulations, not to be used with the flight computer) in
which the timing parameter (second parameter of each command) is in UTC hours.


\item \texttt{lst\_*.sch} - The above UTC schedule file is converted
back into an LST schedule file, given information about the orbit
speed and launch date/time.

\item \texttt{WARNINGS\_readme} - A plain text file containing a list
  of `warnings' encountered during the schedule file simulation. The
  following warning types are implemented:
  \begin{itemize}
  \item 1: Expected duration of command is longer than time to next command.
  \item 2: Expected duration of command is (more than 10 percent) shorter than time to next command.
  \item 3: Command exceeding max allowed elevation.
  \item 4: Command Below min allowed elevation.
  \item 5: ANTI-SUN WARNING: DAY TIME. During the command the boresight violates the anti-sun constraint.
  \item 6: ANTI-MOON WARNING. During the command the boresight violates the anti-moon constraint.
  \end{itemize}

  A typical warning might look like this:

\texttt{5 120 60.8565 57.9042 0.0000 \#  [ ANTI-SUN WARNING: DAY TIME ]}

The first column is the warning type, as defined above; The second
column gives the schedule file command to which the warning applies (1
corresponds to the first command in the schedule file); The next three
numbers are any particular quantities relating to the warning (in this
case the azimuth distance [deg] from the anti-sun direction at the
beginning and end of the command); Finally, a comment follows after
the \# sign.

\end{itemize}

\subsection{Dirfile format and units}

The `full sample rate' pointing dirfile has the \texttt{format} file
shown just below which lists the dirfile entries and their level of
precision. \texttt{AZ} and \texttt{EL} are telescope azimuth and
elevation in degrees (inclusive of any azimuth and elevation
pendulations if enabled). \texttt{LST} is local sidereal time in
hours. \texttt{LON} and \texttt{LAT} are gondola longitude and
latitude in degrees. \texttt{RA}, \texttt{DEC}, \texttt{BETA} and
\texttt{ROLL} are telescope Right Ascension, Declination, orientation
and Roll pendulations, all in radians. \texttt{RJD} is a convention
for a Reduced Julian Day measured in days from an offset
\texttt{RJD0=2454900} defined both in
\texttt{src/utilities\_pointing.h} and in
\texttt{pro/hexsky/define\_constants.pro}.


\verbatiminput{format}


In addition to these entries, the `slow sample rate' dirfile contains
extra entries (written by the IDL code) which describe
\texttt{AZ\_SUN},\texttt{EL\_SUN},\texttt{AZ\_MOON} and
\texttt{EL\_MOON}, which give the azimuth and elevation in degrees of
the Sun and Moon.


\bibliographystyle{aa}
\bibliography{hexsky}

\end{document}
